\documentclass[12pt]{article}

\usepackage{fullpage}
\usepackage{multicol,multirow}
\usepackage{tabularx}
\usepackage{ulem}
\usepackage[utf8]{inputenc}
\usepackage[russian]{babel}
\usepackage{minted}

% Оригиналный шаблон: http://k806.ru/dalabs/da-report-template-2012.tex

\begin{document}
\begin{titlepage}
\begin{center}
\textbf{МИНИСТЕРСТВО ОБРАЗОВАНИЯ И НАУКИ РОССИЙСОЙ ФЕДЕРАЦИИ
\medskip
МОСКОВСКИЙ АВЦИАЦИОННЫЙ ИНСТИТУТ
(НАЦИОНАЛЬНЫЙ ИССЛЕДОВАТЬЕЛЬСКИЙ УНИВЕРСТИТЕТ)
\vfill\vfill
{\Huge ЛАБОРАТОРНАЯ РАБОТА №1} 
по курсу объектно-ориентированное программирование
I семестр, 2019/20 уч. год}
\end{center}
\vfill

Студент \uline{\it {Попов Данила Андреевич, группа 08-208Б-18}\hfill}

Преподаватель \uline{\it {Журавлёв Андрей Андреевич}\hfill}

\vfill
\end{titlepage}

\subsection*{Условие}

Задание №1: написать класс, который реализует комплексное число в алгебраической форме со следующими функциями: 
\begin{enumerate}
\item Сложение
\item Вычитание
\item Умножение 
\item Деление
\item Сравнение
\item Сопряжённое число
\end{enumerate}

\subsection*{Метод решения}


Общее описание алгоритма решения задачи, архитектуры программы и
т.\,п. Полностью расписывать алгоритмы необязательно, но в общих чертах
описать нужно. Приветствуются ссылки на внешние источники,
использованные при подготовке (книги, интернет-ресурсы). 

\subsection*{Описание программы}

Код программы состоит из 3-х файлов:
\begin{enumerate}
\item app/main.cpp: файл, содержащий точку входа приложения
\item include/complex.hpp: файл, содержащий объявление и реализацию inline-функций
\item src/lib/complex.cpp: реализация не-inline методов класса Complex.h
\end{enumerate}

\subsection*{Дневник отладки}

Была одна очень забавная проблема в test.py. После считывания тестового запроса и отправки его в программу, весь тест зависал.
Как оказалось, проблема была в отсутствующем '\\n' в конце запроса, который породил немало головной боли :)

\subsection*{Недочёты}

Метод Str() работает не с внешним буффером, а создаёт каждый раз минимум два:
\begin{enumerate}
\item Для std::stringstream объекта, который предоставляет удобный интерфейс приведения стандартных типов к строке.
\item Для std::string объекта, который является возвращаемым значением.
\end{enumerate}
Данный метод может использовать достаточно много процессорного времени при приведении объектов типа Complex к строке.

\subsection*{Выводы}

В целом, повторил синтаксис написания классов в C++, изучил базовое межпроцессорное взаимодействие через пайпы и отловил один коварный баг,
который не так просто обнаружить. Так же вспомнил, что именно делает модификатор inline, который используется по умолчанию при объявлении с реализацией методов в классе.
\vfill

\subsection*{Исходный код}

{\Huge Complex.hpp}
\inputminted
{C++}{include/complex.hpp}
\pagebreak

{\Huge Complex.cpp}
\inputminted
{C++}{src/lib/complex.cpp}
\pagebreak

{\Huge main.cpp}
\inputminted
{C++}{apps/main.cpp}
\pagebreak

\end{document}
